\documentclass{jsarticle}
\usepackage{amsmath,amssymb}
\usepackage{bm}
\usepackage{graphicx}
\usepackage{ascmac}
\usepackage{url}
\usepackage[]{multicol}
\usepackage{comment}


%
\setlength{\textwidth}{\fullwidth}
\setlength{\textheight}{40\baselineskip}
\addtolength{\textheight}{\topskip}
\setlength{\voffset}{-0.2in}
\setlength{\topmargin}{10pt}
\setlength{\headheight}{10pt}
\setlength{\headsep}{10pt}
%

%% \maketitleの余白を調整
\makeatletter
\renewcommand{\@maketitle}{\newpage
% \null
% \vskip 2em
\begin{center}
{\LARGE \@title \par} \vskip 1.5em {\large \lineskip .5em
\begin{tabular}[t]{c}\@author
\end{tabular}\par}
}
\makeatother
\pagestyle{myheadings}
\markboth{巡検に弓を導入できるか 牛飼}{GEOID \textbf{10}. (2021)}

\title{巡検に弓を導入できるか}
\author{巡検学部鉱物学科2年 牛飼}


\begin{document}
  % \columnseprule=0.3mm
   \maketitle %タイトルを表示するコマンド

   \begin{multicols}{2}
      \newpage
      \section{はじめに}
      文章において、冒頭で内容の概略や背景について述べ、読者が内容になじみやすくするために書かれた部分。本記事でいえば、本節の前に置かれた部分である。導入部は「紹介文」「序文」「序説」「プロローグ(英: prologue)」などとして記載されることもある。\cite{intro}
      \section{手法}
      引用\cite{inyo}
      \subsection{hoge}
      \subsubsection{hogehoge}
      ああああああああああああああああああああああああああああああああああああああああああああああああああああああああああああああああああああああああああああああああああああああああああああああああああああああああああああああああああああああああああああああああああああああああああああああああああああああああああああああああああああああああああああああああああああああああああああああああああああああああああああああああああああああああああああああああああああああああああああああああああああああああああああああああああああああああああああああああああああああああああああああああああああああああああああああああああああああああああああああああああああああああああああああああああああああああああああああああああああああああああああああああああああああああああああああああああああああああああああああああああああああああああああああああああああああああああああああああああああああああああああああああああああああああああああああああああああああああああああああああああああああああああああああああああああああああああああああああああああああああああああああああああああああああああああああああああああああああああああああああああああああああああああああああああああああああああああああああああああああああああああああああああああああああああああああああああああああああああああああああああああああああああああああああああああああああああああああああああああああああああああああああああああああああああああああああああああああああああああああああああああああああああああああああああああああああああああああああああああああああああああああああああああああああああああああああああああああああああああああああああああああああああああああああああああああああああああああああああああああああああああああああああああああああああああああああああああああああああああああああああああああああああああああああああああああああああああああああああああああああああああああああああああああああああああああああああああああああああああああああああああああああああああああああああああああああああああああああああああああああああああああああああああああああああああああああああああああああああああああああああああああああああああああああああああああああああああああああああああああああああああああああああああああああああああああああああああああああああああああああああああああああああああああああああああああああああああああああああああああああああああああああああああああああああああああああああああああああああああああああああああああああああああああああああああああああああああああああああああああああああああああああああああああああああああああああああああああああああああああああああああああああああああああああああああああああああああああああああああああああああああああああああああああああああああああああああああああああああああああああああああああああああああああああああああああああああああああああああああああああああああああああああああああああああああああああああああああああああああああああああああああああああああああああああああああああああああああああああああああああああああああああああああああああああああああああああああああああああああああああああああああああああああああああああああああああああああああああああああああああああああああああああああああああああああああああああああああああああああああああああああああああああああああああああああああああああああああああああああああああああああああああああああああああああああああああああああああああああああああああああああああああああああああああああああああああああああああああああああああああああああああああああああああああああああああああああああああああああああああああああああああああああああああああああああああああああああああああああああああああああああああああああああああああああああああああああああああああああああああああああああああああああああああああああああああああああああああああああああああああああああああああああああああああああああああああああああああああああああああああああああああああああああああああああああああああああああああああああああああああああああああああああああああああああああああああああああああああああああああああああああああああああああああああああああああああああああああああああああああああああああああああああああああああああああああああああああああああああああああああああああああああああああああああああああああああああああああああああああああああああああああああああああああああああああああああああああああああああああああああああああああああああああああああああああああああああああああああああああああああああああああああああああああああああああああああああああああああ

      %bibtexを使いたい時 自分でどうにかして
      \bibliographystyle{sort} 
      \bibliography{mybib}


      %自分で書く時

      \begin{thebibliography}{99}
         \bibitem{gaiyo} Weblio辞典 「概要」\url{https://www.weblio.jp/content/%E6%A6%82%E8%A6%81}
         \bibitem{intro} Wikipedia 「導入部」\url{https://ja.wikipedia.org/wiki/%E5%B0%8E%E5%85%A5%E9%83%A8}
         \bibitem{inyo} Yabe, H. and Shimizu. S., 1925, A new Cretaceous ammonite, \textit{ Crioceras ishiwarai}, from Oshima, province of Rikuzen.
         \textit{Japan. J. Geol. Geogr.}, \textbf{4}, 85-87
       \end{thebibliography}
      \newpage
   \end{multicols}


\end{document}